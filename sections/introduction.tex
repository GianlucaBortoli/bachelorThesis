%!TEX root=../thesis.tex
\chapter{Introduction}\label{chp:introduction}

% + what is real-time
% + difference between soft, hard & firm
% + where we can fint them in everyday life

The term real-time is used to describe any information processing system which has to respond to externally generated input stimuli within a finite and specified period called \emph{deadline}.
This time constraint can be either \emph{hard}, \emph{soft} or \emph{firm}.\\
A deadline is said to be hard if missing it will cause the system to crash or fail unespectedly. This is the case of safety-critical applications and other systems in which the result of an operation must be given within this period of time.\\
For example the microprocessor which is the core part of the control unit of our cars must first analyse the data coming from the sensors around the body and the other parts of the car and then take decisions in a very short period. If it takes too much time to evaluate the distance measured by the parking sensors, alerting the driver when he has already hit a wall would not be so useful.\\
On the other hand, there are situation in which deadlines are not such strict. The usefulness of the result degrades as the time passes after the deadline and thus the quality of service (QoS), but it does not lead to any type of failure or damage.\\
Moreover, it is possible to speak of \emph{probabilistic deadlines} for what concerns soft real-time systems. This term is highly related to the QoS and it means that the deadline must be met with a given probability. In this sense, classic hard-real time systems are considered as a particular case of the soft ones, in which such probability is always zero.\\ 
Lastly, a deadline is said to be firm if a result computed late is completely useless (like hard real-time), leading to a reduction of the quality of service (as it happens with soft real-time) but not to a crash. This type of system can be seen as a sort of mix between the two previous ones.\\
In this thesis the focus lays on the second category.\\
This branch of computer science is continuously gaining popularity, because the number of devices which embed a microprocessor and, in many cases, also a network interface is growing exponentially. Let us think not only to the examples mentioned before, but also to all the electrical applicances we have in our houses: all these tools are becoming way much smarter than their predecessors, building what is called the Internet of Things.    