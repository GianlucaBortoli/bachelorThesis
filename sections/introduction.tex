%!TEX root=../thesis.tex
\chapter{Introduction}\label{chp:introduction}

% + what is real-time
% + difference between soft, hard & firm
% + where we can fint them in everyday life

Real-time systems \cite{hardrealtime} is used to describe any information processing system that has to respond to externally generated input stimuli within a finite amount of time. This constraint is called \emph{deadline} and it can be either hard, soft or firm.\\
A deadline is said to be \emph{hard} if a deadline miss causes a critical failure in the system. This is the case of safety-critical applications and other systems in which the result of an operation must be given within the deadline. For example, if the microprocessor, that is the core part of the control unit of our cars, takes too much time to evaluate the distance measured by the parking sensors it will lead to a crash. This event is not admissible since it jeopardizes the driver's safety.\\
On the other hand, \emph{soft} deadlines describe situations in which deadlines are not such strict. The usefulness of the result degrades as time passes after the deadline. The same is valid for the Quality of Service (QoS), but it does not lead to any type of failure or damage. For example, most of the multimedia streaming services are based on this type of deadline, given that the tasks taken into consideration do not put any life at risk.\\
Finally, a deadline is said to be \emph{firm} if a result computed late is completely useless (like hard real-time), leading to a reduction of the QoS (as it happens with soft real-time) but not to a crash. This types of systems can be seen as a sort of mix between the two previous ones.\\
Moreover, it is possible to talk of \textbf{probabilistic deadlines} \cite{abeni} for what concerns \emph{soft} real-time systems. This term is highly related to the QoS and it indicates that the deadline would be met with a given probability. In this sense, deterministic deadlines are considered as a particular case of the probabilistic ones in which the deadline miss probability is zero.\\ 
In this thesis the focus lays on \textbf{soft real-time} tasks. This branch of computer science is continuously gaining popularity, because of the number of embedded devices which have a microprocessor and, in many cases, also a network interface is growing exponentially. In addition to the examples mentioned before, all the electrical applicances we have in our houses are becoming way much smarter than their predecessors, building what is called the Internet of Things.\\
This thesis is structured as follows. In Chapter 2 a formal definition of the problem is given. Chapter 3 presents the PROSIT tool. Chapter 4 is devoted to display some experimental results, while Chapter 5 concludes the thesis.      