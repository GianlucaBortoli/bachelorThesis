%!TEX root=thesis.tex
\usepackage[
  pdftex,
  plainpages=false,
  colorlinks,
  hyperindex,
  pdfpagemode=UseNone,
  bookmarksopen,
  linkcolor=black,
  citecolor=black,
  urlcolor=black]{hyperref}
\usepackage{graphicx}
\usepackage[boxed]{algorithm}
\usepackage{setspace}
\usepackage{algorithm}
\usepackage{algorithmic}
\usepackage{subfigure}
\usepackage[utf8]{inputenc}
\usepackage[nocompress]{cite}
\usepackage[show]{chato-notes}
\usepackage{booktabs}
\usepackage{enumitem}

\usepackage{url}
\hypersetup{
    unicode=true,          % non-Latin characters in Acrobats bookmarks
    pdftoolbar=true,        % show Acrobats toolbar?
    pdfmenubar=true,        % show Acrobats menu?
    pdffitwindow=false,     % window fit to page when opened
    pdfstartview={FitH},    % fits the width of the page to the window
    pdftitle={My title},    % title
    pdfauthor={Author},     % author
    pdfsubject={Subject},   % subject of the document
    pdfcreator={Creator},   % creator of the document
    pdfproducer={Producer}, % producer of the document
    pdfnewwindow=true,      % links in new window
    colorlinks=true,       % false: boxed links; true: colored links
    linkcolor=black,          % color of internal links
    citecolor=black,        % color of links to bibliography
    filecolor=black,      % color of file links
    urlcolor=black           % color of external links
}
\newcommand{\spara}[1]{\smallskip\noindent{\bf #1}}
\newcommand{\mpara}[1]{\medskip\noindent{\bf #1}}
\newcommand{\para}[1]{\noindent{\bf #1}}

\newtheorem{definition}{Definition}
\newtheorem{proposition}{Proposition}
\newtheorem{property}{Property}
\newtheorem{theorem}{Theorem}
\newtheorem{corollary}{Corollary}
\newtheorem{claim}{Claim}
\newtheorem{example}{Example}
\newtheorem{lemma}{Lemma}
\newtheorem{problem}{Problem}
\newtheorem{assumption}{Assumption}
\newtheorem{observation}{Observation}



\newcommand{\Abs}[1]{\left|#1\right|}
\newcommand{\Tuple}[1]{\left<#1\right>}
\newcommand{\Set}[1]{\left\{#1\right\}}
\newcommand{\List}[1]{\left[#1\right]}
\newcommand{\Paren}[1]{\left(#1\right)}
\newcommand{\Binom}[2]{\left(#1 \atop #2\right)}
\newcommand{\Floor}[1]{\left\lfloor #1 \right\rfloor}
\newcommand{\Ceil}[1]{\left\lceil #1 \right\rceil}
\newcommand{\IntO}[1]{\left(#1\right)}
\newcommand{\IntC}[1]{\left[#1\right]}
\newcommand{\IntLO}[1]{\left(#1\right]}
\newcommand{\IntRO}[1]{\left[#1\right)}
\newcommand{\Choose}[2]{\left( #1 \atop #2\right)}

\newcommand{\sharpP}{{\bf \#P}}
\newcommand{\NP}{$\mathbf{NP}$}
\newcommand{\NPhard}{$\mathbf{NP}$-hard}
\newcommand{\NPcomplete}{$\mathbf{NP}$-complete}
\newcommand{\SPcomplete}{$\mathbf{\#P}$-complete}
\newcommand{\SPhard}{$\mathbf{\#P}$-hard}

\renewcommand{\algorithmicrequire}{\textbf{Input:}}
\renewcommand{\algorithmicensure}{\textbf{Output:}}
\renewcommand{\vec}[1]{\mathbf{#1}}

\usepackage{amssymb}% http://ctan.org/pkg/amssymb
\usepackage{pifont}% http://ctan.org/pkg/pifont
\newcommand{\cmark}{\ding{51}}%
\newcommand{\xmark}{\ding{55}}%

\usepackage{listings} % Source code support
\usepackage{etoolbox}

\lstset{ %
%  backgroundcolor=\color{myBlack},   % choose the background color; you must
  % add \usepackage{color} or \usepackage{xcolor}
  basicstyle=\ttfamily\footnotesize\color{black},        % the size of the fonts
  % that are used for the code
  breakatwhitespace=false,         % sets if automatic breaks should only happen at whitespace
  breaklines=true,                 % sets automatic line breaking
  captionpos=b,                    % sets the caption-position to bottom
%  commentstyle=\color{grey},       % comment style
  deletekeywords={...},            % if you want to delete keywords from the given language
  escapeinside={\%*}{*)},          % if you want to add LaTeX within your code
  extendedchars=true,              % lets you use non-ASCII characters; for 8-bits encodings only, does not work with UTF-8
  frame=single,                    % adds a frame around the code
  keepspaces=true,                 % keeps spaces in text, useful for keeping indentation of code (possibly needs columns=flexible)
  keywordstyle=\color{red},       % keyword style
 % language=go,                 % the language of the code
  morekeywords={*,...},            % if you want to add more keywords to the set
  numbers=left,                    % where to put the line-numbers; possible values are (none, left, right)
  numbersep=5pt,                   % how far the line-numbers are from the code
 % numberstyle=\tiny\color{myGray}, % the style that is used for the
  % line-numbers
%  rulecolor=\color{white},         % if not set, the frame-color may be changed on line-breaks within not-black text (e.g. comments (green here))
  showspaces=false,                % show spaces everywhere adding particular underscores; it overrides 'showstringspaces'
  showstringspaces=false,          % underline spaces within strings only
  showtabs=false,                  % show tabs within strings adding particular underscores
  stepnumber=1,                    % the step between two line-numbers. If it's 1, each line will be numbered
  stringstyle=\color{green},     % string literal style
  tabsize=2,                       % sets default tabsize to 2 spaces
  title=\lstname                   % show the filename of files included with \lstinputlisting; also try caption instead of title
}

\usepackage[toc,page]{appendix}		% appendice
\usepackage{lscape}
